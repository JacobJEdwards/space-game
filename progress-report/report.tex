\documentclass[final]{cmpreport}

\usepackage[newfloat]{minted}
\usepackage{rotating}
\usepackage{subfloat}
\usepackage{color}
\usepackage{pdfpages}
\usepackage{enumitem}
\usepackage[utf8]{inputenc}
\usepackage{textgreek}

\newenvironment{code}{\captionsetup{type=listing}}{}
\SetupFloatingEnvironment{listing}{name=Source Code}

\title{Lost in Space}

\author{Jacob Edwards}
\registration{100389703}
\supervisor{Yingliang Ma}
\ccode{CMP-6056B}
\summary{This report details the current development of a 3D space shooter game, Lost in Space, using the Unity game engine.}

\onePageLists

\begin{document}
    \section{Introduction}\label{sec:introduction}

    \section{Background / Similar games}\label{sec:background}

    \section{Design}\label{sec:design}

    \section{Game Features}\label{sec:features}

    The following key game mechanics are implemented in the game:

    \subsection{Space Movement and Combat}\label{subsec:movement}

    The gameplay features 3 major states - Zero G player movement, Spaceship entry, flying and shooting, and terrestrial
    movement.
    This subsection will cover the first two states, with the third being covered in the next subsection.

    The player, when not in their ship is able to move in 3D world space, with the ability to move in any direction and
    rotate in any direction.
    This is done using Unity's InputSystem\citep{unity:inputsystem} to get the player's input and apply it to the player's
    Rigidbody component\citep{unity:rigidbody} using thrust and torque force, as well as gliding to simulate the lack of
    friction in space.

    From this state, the player can enter their ship by looking at it and pressing \textit{F}, which is the interact
    key in the game.
    A ray is shot from the player's camera to forwards, and if its hits an interactable object (such as the player's ship),
    the player is able to interact, where the interaction function for the ship causes the player to enter the ship and
    switch to the spaceship state, where the player is now controlling the ship.

    The ship movement is very similar to the player movement, with the player able to move in 3D space, rotate in any
    direction, but with a few key differences:
    The thrust is much greater in the forwards direction, simulating the fact that the boosters are at the back, and
    the pitch and yaw movement is much slower, simulating the fact that the ship is much larger and heavier than the
    player, as well as requiring more force to rotate.

    From the ship, the player is able to shoot lasers, which are raycast from the ship's front, and if they hit a
    damageable object (such as an asteroid), they will deal damage to it.
    If enough damage is dealt, the object will be destroyed.

    \subsection{Planetary Exploration}\label{subsec:planetary}

    From the ship, the player is able to land on planets or large asteroids, where they can exit the ship and explore
    the surface with gravity, requiring a change to the player's movement to simulate the gravity, with the addition of
    a jump mechanic and more fluid pitch and yaw movement.
    A jetpack is also available allowing the player to fly within this gravity affected environment.

    As well as being able to land the ship on the planet, if the player is in Zero G movement and gets close enough
    to a planet, they will be affected by the gravity.

    As the gravity is custom, a custom movement system was implemented to allow for more control over the player's
    movement.

    \appendix

    \section{Screen Captures}\label{sec:screen-captures}

    \section{Code}\label{sec:code}
\end{document}
